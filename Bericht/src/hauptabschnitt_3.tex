\section{Hauptabschnitt Drei}\label{hauptabschnitt3}
\Blindtext

% beispiel für quellcode listings
\lstset{language=xml}
\begin{lstlisting}[float=ht!, frame=htrbl, caption={die datei {\normalfont \ttfamily  data-config.xml} dient als beispiel für xml quellcode}, label={lst:dataconfigxml}]
<dataconfig>
  <datasource type="jdbcdatasource" 
              driver="com.mysql.jdbc.driver"
              url="jdbc:mysql://localhost/bms_db"
              user="root" 
              password=""/>
  <document>
    <entity name="id"
        query="select id, htmlbody, sentdate, sentfrom, subject, textbody
        from mail">
    <field column="id" name="id"/>
    <field column="htmlbody" name="text"/>
    <field column="sentdate" name="sentdate"/>
    <field column="sentfrom" name="sentfrom"/>
    <field column="subject"  name="subject"/>
    <field column="textbody" name="text"/>
    </entity>
  </document>
</dataconfig>
\end{lstlisting}
\Blindtext

\subsection{Unterabschnitt von Drei}\label{hauptabschnitt3.1}

\Blindtext

\lstset{language=java}
\begin{lstlisting}[float=ht!, frame=htrbl, caption={das listing zeigt java quellcode}, label={lst:result2}]
import java.io.*;
import javax.servlet.*;
import javax.servlet.http.*;

public class HelloServlet extends HttpServlet {

  protected void doGet(HttpServletRequest  request, 
      HttpServletResponse response)
      throws IOException, ServletException {

      response.setContentType("text/html");
      PrintWriter out = response.getWriter();
      out.println("<!doctype html><html>");
      out.println("<head> <meta charset='utf-8'>");
      out.println("<title>webapp</title> </head>");
      out.println("<body>Hello</body>");
      out.println("</html>");
      ServletContext context = getServletContext();
      context.log("simple logging");
  }
}
\end{lstlisting}
\Blindtext

\subsection{Neuer Unterabschnitt von Drei}\label{hauptabschnitt3.2}

\Blindtext
