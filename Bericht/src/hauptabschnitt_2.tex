\section{Hauptabschnitt Zwei}\label{hauptabschnitt}

\Blindtext
\Blindtext

\subsection{Unterabschnitt von Zwei}\label{unterabschnitt_1}

\Blindtext
\Blindtext
\Blindtext
% Beispiel: Tabelle 
\begin{table}[!ht]
  \centering
    \caption[Beispieltabelle]{Beispieltabelle~\cite[S.400]{KnutThea2009}}
    \label{table:Beispieltabelle}
    \begin{tabular}{ | l | c | }
      \hline
      Überschrift 1 & Überschrift 2 \\ \hline 
      Info 1 & Info 2 \\ \hline
      Info 3 & Info 4 \\ \hline
      \hline
      \multicolumn{2}{|c|}{Info in einer Zelle} \\
      \hline
    \end{tabular}
  \end{table}
\Blindtext


Lorem ipsum dolor sit amet, consetetur \ac{eu} sadipscing elitr, sed diam nonumy eirmod tempor invidunt ut labore et dolore magna aliquyam erat, sed diam voluptua. At vero eos et accusam et justo duo dolores et ea rebum. Stet clita 
kasd gubergren, no sea takimata sanctus est Lorem ipsum dolor sit amet. \ac{eu} Lorem ipsum dolor sit amet, consetetur sadipscing elitr, sed diam nonumy eirmod tempor invidunt ut labore et dolore magna aliquyam erat, sed diam voluptua. 
At vero eos et accusam et justo duo dolores et ea rebum. Stet clita kasd gubergren, no sea takimata \ac{https} sanctus est Lorem ipsum dolor sit amet.

\subsection{Neuer Unterabschnitt von Zwei}\label{unterabschnitt_2}

\Blindtext
  \begin{table}[!ht]
    \centering
    \caption{Beispieltabelle 2}
    \label{table:Beispieltabelle_2}
  \begin{tabular}{ccc}\toprule
    A&B&C \\ \midrule
    a&b&c \\ \cmidrule{1-3}
    1&2&3\\ \bottomrule
    \end{tabular}
  \end{table}
\Blindtext
\Blindtext



  
\subsection{Neuer Unterabschnitt von Zwei}\label{unterabschnitt_3}

\begin{quote}
  \singlespacing \small
  "`Lorem ipsum dolor sit amet, consectetur adipiscing elit, sed do eiusmod tempor incididunt ut labore et dolore magna aliqua. Et odio pellentesque diam volutpat commodo sed. Donec pretium vulputate sapien nec 
  sagittis aliquam. Nullam ac tortor vitae purus faucibus."'~\cite[S. 189]{KnutThea2009}
\end{quote}

\Blindtext

\subsubsection{Unterabschnitt von Unterabschnitt von Zwei}\label{unterunterabschnitt}
\Blindtext
\Blindtext
% Beispiel für Formeln
Die Zuordnung aller möglichen Werte, welche eine Zufallsvariable annehmen kann nennt man \emph{Verteilungsfunktion} von $X$.

\begin{quote}
Die Funktion F: $\mathbb{R} \rightarrow$ [0,1] mit $F(t) = P (X \le t)$ heißt Verteilungsfunktion von $X$. vgl.~\cite[S.55]{mf2005}
\end{quote}

\begin{quote}
Für eine stetige Zufallsvariable $X: \Omega \rightarrow \mathbb{R}$ heißt eine integrierbare, nichtnegative reelle Funktion 
$w: \mathbb{R} \rightarrow \mathbb{R}$ mit $F(x) = P(X \le x) = \int_{-\infty}^{x} w(t)dt$ die \emph{Dichte} oder \emph{Wahrscheinlichkeitsdichte} 
der Zufallsvariablen $X$. vgl.~\cite[S.225-323]{KnutThea2009}
\end{quote}

\Blindtext

\subsubsection{Neuer Unterabschnitt von Unterabschnitt von Zwei}\label{neuerunterunterabschnitt}

\Blindtext
