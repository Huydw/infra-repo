%%%%%%%%%%%%%%%%%%%%%%%%%%%%%%%%%%%%%%%%%%%%%%%%%%%%%%%%%%%%%%%%%%%
%                                                                 %                    
%                 Packages / Grundeinstellungen                   %
%                                                                 %
%%%%%%%%%%%%%%%%%%%%%%%%%%%%%%%%%%%%%%%%%%%%%%%%%%%%%%%%%%%%%%%%%%%


% Festlegung des Allgemeinen Dokumentenformat
\documentclass[a4paper,12pt,parskip=half,headsepline,DIV=12,numbers=noenddot]{scrartcl}

%BCOR12mm, Korrektur fuer die Bindung
%DIV12 DIV-Wert fuer die Erstellung des Satzspiegels
%numbers=noenddot

% Keine floats in andere Sections
\usepackage[section]{placeins}

% Weitere Pakete
\usepackage{microtype}
\usepackage{scrhack}
\usepackage{caption}
\usepackage{fontspec}

% Booktabs Tabellen
\usepackage{booktabs}

% Grafiken aus PNG Dateien einbinden
\usepackage{graphicx}

% Deutsche Sonderzeichen und Silbentrennung nutzen
\usepackage[ngerman]{babel}
\usepackage{blindtext}

% Eurozeichen einbinden
\usepackage[right]{eurosym}

% Kopf- und Fußzeilen
\usepackage[headsepline,autooneside=false]{scrlayer-scrpage}
\clearpairofpagestyles

% Zeichenencoding
\usepackage[T1]{fontenc}

% Schriftart 
\usepackage{lmodern}

% Floatende Bilder ermöglichen
\usepackage{floatflt}

% Für Tabellen
\usepackage{array}

% Mehrseitige Tabellen ermöglichen
\usepackage{longtable}

% Unterstützung für Schriftarten
\newfontfamily\popp{Poppins}
\fontspec{Poppins}

% Paket für Boxen im Text
\usepackage{fancybox}

% Bricht lange URLs "schön" um
\usepackage[hyphens,obeyspaces,spaces]{url}

% Paket für Textfarben
\usepackage{color}

% Mathematische Symbole importieren
\usepackage{amssymb}

% Erzeugt Inhaltsverzeichnis mit Querverweisen zu den Abschnitten (PDF Version)
\usepackage[bookmarksnumbered,hyperfootnotes=false]{hyperref}
\hypersetup{
     colorlinks=true,
     linkcolor=black,
     filecolor=blue,
     citecolor = black,      
     urlcolor=blue,
}

% Festlegung Art der Zitierung - Numerisch (IEEE) 
\bibliographystyle{IEEEtran}
% Festlegung Art der Zitierung - APA (apalike-german) 
%\bibliographystyle{apalike-german}

% Paket für Zeilenabstand
\usepackage{setspace}

% Für Bildbezeichner
\usepackage{capt-of}

% Für Stichwortverzeichnis
\usepackage{makeidx}

% Konfiguriere das Inhaltsverzeichnis
\usepackage{tocbasic}
\DeclareTOCStyleEntries[
  raggedentrytext,
  %numwidth=0pt, if numbers=noenddot is not set
  numsep=1ex,
  dynnumwidth,
]{tocline}{chapter,section}
\DeclareTOCStyleEntries[
  linefill=\TOCLineLeaderFill,
]{tocline}{section,subsection,subsubsection,paragraph,subparagraph}

% Für Listings
\usepackage{listings}
\lstset{numbers=left, numberstyle=\tiny, numbersep=5pt, keywordstyle=\color{black}\bfseries, stringstyle=\ttfamily,showstringspaces=false,basicstyle=\footnotesize,captionpos=b, breaklines=true}

% Indexerstellung
\makeindex

% Abkürzungsverzeichnis
\usepackage[printonlyused, smaller, withpage]{acronym}

% Schriftart Helvetica verwenden
%\usepackage{helvet}
%\renewcommand\familydefault{\sfdefault}

\hypersetup{pdfinfo={
Title={Titel der Arbeit},
Author={Fabian Lignitz}
}}

%%%%%%%%%%%%%%%%%%%%%%%%%%%%%%%%%%%%%%%%%%%%%%%%%%%%%%%%%%%%%%%%%%%
%                                                                 %                    
%                     Beginn des Inhalts                          %
%                                                                 %
%%%%%%%%%%%%%%%%%%%%%%%%%%%%%%%%%%%%%%%%%%%%%%%%%%%%%%%%%%%%%%%%%%%

%%%%%%%%%%%%%%%%%%%%%%%%%%%%%%%%%%%%%%%%%%%%%%%%%%%%%%%%%%%%%%%%%%%
%  Special Characters:                                            %
%                                                                 %
%             \& \% \$ \# \_ \{ \}                                %
%             \textasciitilde (~)                                 %
%             \textasciicircum (^)                                %     
%             \textbackslash (\)                                  %                    
%                                                                 %
%%%%%%%%%%%%%%%%%%%%%%%%%%%%%%%%%%%%%%%%%%%%%%%%%%%%%%%%%%%%%%%%%%%

\begin{document}

%Definition Header
\automark[subsection]{section}
\KOMAoptions{headsepline=true}
%\ihead{Kopfzeile innen}
%\chead{Kopfzeile außen}
\ohead{\headmark}

%Definition footer
%\ifoot{Fußzeile innen}
%\cfoot{Fußzeile Mitte}
\ofoot{\pagemark}

% hier werden die Trennvorschläge inkludiert
\input{src/basic_structure/trennung.tex}

% Leere Seite am Anfang
%\thispagestyle{empty} % erzeugt Seite ohne Kopf- / Fusszeile
%\mbox{}
%\newpage

% Titelseite %
%%%%%%%%%%%%%%%%%%%%%%%%%%%%%%%%%
%           Deckblatt           %
%%%%%%%%%%%%%%%%%%%%%%%%%%%%%%%%%

\thispagestyle{empty}
\begin{figure}[h!]
 \centering
 \includegraphics[width=0.6\textwidth]{src/abbildungen/logoneu.png}
\end{figure}
\begin{center}
\large\textbf{Fachbereich II \\ Management und Informationssysteme}\\
\large\textbf{Informatik}\\
\vspace{1cm}
%%%%%%%%%%%%%%%%%%%%%%%%%%%%%Für Hausarbeit%%%%%%%%%%%%%%%%%%%%%%%%%%%%%%%%%%%
\large\textbf{Modul\\ Infrastruktur}\\
%%%%%%%%%%%%%%%%%%%%%%%%%%%%%%%%%%%%%%%%%%%%%%%%%%%%%%%%%%%%%%%%%%%%%%%%%%%%%%
%%%%%%%%%%%%%%%%%%%%%%%%%%%Für Bachelorarbeit%%%%%%%%%%%%%%%%%%%%%%%%%%%%%%%%%
%\LARGE\textbf{Bachelorarbeit}\\
%\large{zur Erlangung des akademischen Grades \\ Bachelor of Science}\\
%%%%%%%%%%%%%%%%%%%%%%%%%%%%%%%%%%%%%%%%%%%%%%%%%%%%%%%%%%%%%%%%%%%%%%%%%%%%%%
\vspace*{\fill}
\line(1,0){450}\\
\doublespacing
\textbf{\Large{Bericht zu Infrastruktur}}\\
% \textbf{\large{Untertitel}}\\
\line(1,0){450}\\
\end{center}
\vspace*{\fill}
\onehalfspacing
\begin{flushleft}
\begin{tabular}{llll}
\textbf{Vorgelegt von:} & & Phan Huy Tran MatNr. 39142 & \\
\textbf{Vorgelegt am:} & & \today &\\
%%%%%%%%%%%%%%%%%%%%%%%%%%%%%Für Hausarbeit%%%%%%%%%%%%%%%%%%%%%%%%%%%%%%%%%%%
\textbf{Dozent:in:} & & Prof. Dr.-Ing. Oliver Radfelder  & \\
%%%%%%%%%%%%%%%%%%%%%%%%%%%%%%%%%%%%%%%%%%%%%%%%%%%%%%%%%%%%%%%%%%%%%%%%%%%%%%
%%%%%%%%%%%%%%%%%%%%%%%%%%%Für Bachelorarbeit%%%%%%%%%%%%%%%%%%%%%%%%%%%%%%%%%
%\textbf{Erstgutachter:} & & Prof. Dr. Maxi Mustermann & \\
%\textbf{Zweitgutachterin:} & & Prof. Dr. Maxi Musterfrau &\\
%%%%%%%%%%%%%%%%%%%%%%%%%%%%%%%%%%%%%%%%%%%%%%%%%%%%%%%%%%%%%%%%%%%%%%%%%%%%%%
\end{tabular}
\end{flushleft}

\newpage

% Singlespacing (Default)
\singlespacing

% Abstract falls gewünscht
%\onehalfspacing
%\thispagestyle{empty}
%\input{abstract}
%\newpage

% Seitenzählung nach dem Inhaltsverzeichnis bei 1 beginnen
%\setcounter{page}{1}

% Inhaltsverzeichnis anzeigen
\thispagestyle{empty}
\tableofcontents
\newpage

% Abbildungsverzeichnis anzeigen
\listoffigures
\newpage
%\addcontentsline{toc}{section}{Abbildungsverzeichnis}

% Tabellenverzeichnis anzeigen
\listoftables
\newpage
%\addcontentsline{toc}{section}{Tabellenverzeichnis}


% Listingverzeichnis anzeigen
\renewcommand{\lstlistlistingname}{Listingverzeichnis}
\lstlistoflistings
\newpage
%\addcontentsline{toc}{section}{Listingverzeichnis}

% Abkürzungsverzeichnis anzeigen
\ohead{Abkürzungsverzeichnis} % Korrektur für Header 
\section*{Abkürzungsverzeichnis}\label{Abkuerzungsverzeichnis}
\input{src/basic_structure/abkuerzungen.tex}
\newpage
%\addcontentsline{toc}{section}{Abkürzungsverzeichnis}

% Header für den Inhalt 
\KOMAoptions{headsepline=true}
\ohead{\headmark}

% Input Inhalt
\section{Einleitung}\label{einleitung}
\Blindtext
% Beispiel für Bildintegration
\begin{figure}[!ht]
 \centering
 \includegraphics[width=0.8\textwidth]{src/abbildungen/logoneu.png}
 \caption[Beschreibung]{Beschreibung~\cite[S.14]{mf2005}}
\label{fig:Beschreibung}
\end{figure}
\Blindtext


\input{src/hauptabschnitt_2.tex}
\input{src/hauptabschnitt_3.tex}
\input{src/fazit.tex}
\newpage

% Literaturverzeichnis anzeigen
\phantomsection
\addcontentsline{toc}{section}{Literaturverzeichnis}
\renewcommand\refname{Literaturverzeichnis}
\bibliography{bibtex/Hauptdatei.bib}
\newpage

% Kein Header für Anhang (Deckblatt) 
\KOMAoptions{headsepline=false}
\ohead{}

% Beginn Anhang
\input{src/anhang/anhang_deckblatt.tex}

% Anhang römisch 
\renewcommand{\thesection}{\Roman{section}} 
\renewcommand{\thesubsection}{\Roman{subsection}}
\setcounter{section}{0}

% Header Anhang (Inhalt)
\KOMAoptions{headsepline=true}
\ohead{\headmark}
\automark{subsection}

% Input Anhang 
\input{src/anhang/anhang.tex}
\newpage

% Eidesstattliche Erklärung
\phantomsection
\addcontentsline{toc}{section}{Eidesstattliche Erklärung}

% Header für Erklärung
\ohead{Eidesstattliche Erklärung}

% Input Erklärung
\input{src/basic_structure/erklaerung.tex}

% Leere Abschlussseite
%\newpage
%\thispagestyle{empty} % erzeugt Seite ohne Kopf- / Fusszeile
%\mbox{}

\end{document}
